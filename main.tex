%!TEX root = <main.tex>
\documentclass[acmsmall]{acmart}

% \copyrightyear{2019} 
% \acmYear{2019} 
% \setcopyright{acmcopyright}
% \acmConference[SIGMOD '19]{2019 International Conference on Management of Data}{June 30-July 5, 2019}{Amsterdam, Netherlands}
% \acmBooktitle{2019 International Conference on Management of Data (SIGMOD '19), June 30-July 5, 2019, Amsterdam, Netherlands}
% \acmPrice{15.00}
% \acmDOI{10.1145/3299869.3319874}
% \acmISBN{978-1-4503-5643-5/19/06}

\settopmatter{printacmref=false}
% \fancyhead{}

\usepackage[font=small,labelfont=bf]{caption}
\usepackage{graphicx,xspace,verbatim,comment}
\usepackage{hyperref,array,color,balance,multirow}
\usepackage{balance,float,url,amsfonts,alltt}
\usepackage{mathtools,rotating,amsmath,amssymb}
\usepackage{color,ifpdf,fancyvrb}
\usepackage{etoolbox,listings,subcaption}
\usepackage{bigstrut,morefloats,pbox}
\usepackage{amsmath}
\usepackage{algorithm}
\usepackage[noend]{algpseudocode}
\usepackage{booktabs}
\usepackage{bm}

\newtheorem{theorem}{Theorem}[section]
\newtheorem{prop}{Proposition}[section]
\newtheorem{corollary}[theorem]{Corollary}
\newtheorem{lemma}[theorem]{Lemma}
\newtheorem{definition}{Definition}[section]

\newcommand{\red}{\textcolor{red}}
\newcommand{\titlename}{Machine Learning for Database Internals: A Survey}

\DeclarePairedDelimiter{\ceil}{\lceil}{\rceil}

\newenvironment{packeditems}{
\begin{itemize}
  \setlength{\itemsep}{1pt}
  \setlength{\parskip}{0pt}
  \setlength{\parsep}{0pt}
}{\end{itemize}}

\newenvironment{packedenums}{
\begin{enumerate}
  \setlength{\itemsep}{1pt}
  \setlength{\parskip}{0pt}
  \setlength{\parsep}{0pt}
}{\end{enumerate}}

\newcolumntype{P}[1]{>{\centering\arraybackslash}p{#1}}

\DeclareMathOperator*{\argmin}{arg\,min}

\pagestyle{empty}  
\pagenumbering{arabic}

\begin{document}\sloppy
\title{\titlename}


\author{Supun Nakandala}
\affiliation{
  \institution{University of California, San Diego}
}
\email{snakanda@eng.ucsd.edu}

\maketitle
\begin{abstract}
\end{abstract}

\section{Introduction}

Machine Learning (ML) is revolutionizing many fields.
It is now being used in a diverse range of domains including e-commerce, web and social media, interactive agents, and even in critical applications in autonomous vehicles and healthcare.
From a software systems point of view, ML can be viewed as a flexible framework for \textit{programming the unprogrammable}.
For instance, it is nearly impossible for a software engineer to program an object detector that can identify different objects in an image.
But using ML and a sufficient amount of labeled training data one can \textit{learn a program} to do the same that can even surpass human-level accuracy.

Database management systems (DBMSs) are complex software systems, that are the result of decades of research and complex software development.
They are at the core of many critical software applications and have also influenced much of the development of other popular data management systems such as NoSQL, Big Data, and Machine Learning Systems.
Out of the many types of database management systems, relational database management systems remains the most widely used type.
Thus for this paper, we primarily focus on relational database management systems.

DBMSs are also full of really-hard unprogrammable problems such as query optimization, physical database design optimization, and buffer management.
These problems are unprogrammable either because they have interactable search spaces and/or due to the inability to know what is going to happen next and hence cannot develop a program to find the optimal solution.
As a result, DBMS developers have to resort to using heuristics (and/or solution space restrictions) in order to solve these problems.
The goal of these heuristics is not to find an optimal solution for a specific instance but to find a solution which is not bad for all instances.

In this paper, we survey the existing landscape of using ML to program the programmable in DBMSs. 
We divide the DBMS into three main sub-components: 1) Query Parser, 2) Relational Engine, 3) Execution Engine, and identify several systems that incorporate ML into optimizing these sub-components. 
We also present several overarching design decisions that one has to make when integrating ML-based components into a DBMS and the trade-offs of available options.
Finally, we identify several open challenges that bottleneck the practical adoption of ML in DBMS internals.
The table provides a summary of this survey.

The rest of this paper is organized as follows.
Section 2 provides some background on ML methods and DBMS internal in order to understand the rest of this paper.
Section 3, 4, and 5 present the use of ML in Query Parser, Relational Engine, and Execution Engine, respectively.
Section 6 presents the design choices and Section 7 presents the open-challenges.
We give our concluding remarks in Section 8.

\section{Background}
\subsection{Database System Architecture}
\subsection{Machine Learning Methods}

\section{ML for Relational Engine}
Relational Engine is one of the most important components in a database management system,  that has been extensively studied for the last 40 years.
We identify three different sub-areas of ML applications in the relational engine: ML for (1) query optimization, (2) physical database design automation, and (3) approximate query processing.
Next, we discuss some of the most prominent works in each of these sub-areas.

\subsection{Query Optimization}
The goal of query optimization is to transform an input relational algebra expression into an optimal query evaluation plan (QEP).
To achieve this, traditional query optimizers perform a search over the space of potential QEPs and pick the one with the least total cost.
However, estimating the total cost is a complex task and it is often approximated by estimating the total size of the intermediate tuples generated during query evaluation.
This is called the \textit{cardinality estimation} problem, which remains still an unsolved problem despite advancements over many decades. 
Database optimizers often make assumptions such as uniformity, independence, and the principle of inclusion~\cite{leis2018query} to perform cardinality estimation.
These assumptions often do not hold in real data and optimizers tend to under- or over-estimate the query cost and pick sub-optimal query evaluation plans, which can be worse by orders of magnitude.
We found that the use of ML for query optimization can be broadly divided into two major approaches.
The first approach is to train ML models for cardinality estimation and integrate them with the existing search strategies in the optimizer.
The second approach is to completely replace the traditional query optimizer by using ML to generate the QEP, end-to-end.

\subsubsection{Cardinality Estimation}
LEO~\cite{leo} is one of the very first systems developed to use ML techniques to improve the cardinality estimation.
It uses the optimizer estimated cardinalities and the cardinalities observed during execution to learn predicate level linear models to predict the correct output cardinalities.
However, not accounting for QEP structure and using the optimizer estimated cardinality as the only feature prevents LEO from achieving high accuracies.

CardLearner~\cite{cardlearner} extends the idea of LEO and builds ML models to predict cardinalities of commonly occurring query templates instead of each predicate.
A template is a family of queries with the same structure but varying parameters and inputs.
Instead of training one single model for all commonly occurring query templates, it trains separate ML models for each template.
CardLerarner uses several hand-engineered features including metadata features, input cardinalities of all input datasets, and features associated with operators.
Operating on query templates enables CardLearner to capture the semantics of the QEP and correlation in data and yield better results compared to predicate level approaches such as LEO.
However, its accuracy degrades when faced with new queries and queries which have unseen sub-graphs.


One could also train ML models to predict the output cardinality of entire QEPs.
% QPPNet~\cite{qppnet}, Naru~\cite{naru}, and DeepDB~\cite{deepdb} are examples of such systems.
QPPNet uses a novel neural network architecture that matches the QEP structure.
It is composed of stacking sub-neural-modules corresponding to each operator in the QEP which takes in hand-engineered features as well as output from other sub-neural-modules.
As a result, it can learn the correlation in data, relationships between operator features, and plan structure to predict the QEP cost more accurately.
However, one of the limitations of replacing just the cardinality estimation component in the optimizer is that the model will have zero knowledge about the plans that were never generated by the optimizer, which limits the optimizer's ability to generate new and better plans.

Thus, instead of learning from query workloads, Naru~\cite{naru}, and DeepDB~\cite{deepdb} try to solve the cardinality estimation problem by modeling the joint probability distribution of the data.
Naru uses deep autoregressive models and combines it with a novel Monte Carlo integration scheme to efficiently support range and wildcard queries.
DeepDB uses Relational Sum-Product Networks (RSPNs), a variant of a probabilistic graphical model, to model the joint probability distribution.

It is important to mention that some of the above systems (e.g., LEO in IBM DB2 and CardLearner in Microsoft SCOPE) have been (or are being) used in enterprise systems.

\subsubsection{Query Evaluation Plan Generation.} The QEP search strategy in the optimizer closely resembles the reinforcement learning (RL) methods in machine learning.
Hence, several systems have tried to replace the entire QEP generation process using RL instead of using ML models to augment an existing optimizer.
This ability to learn from the feedback from the chosen QEPs enables RL models to avoid choosing the same bad QEP over and over again.

SkinnerDB~\cite{skinnerdb} proposes an intra-query regret-bounded RL learning strategy to find the optimal join ordering for a QEP.
Instead of learning from past query executions, it uses UTC algorithm~\cite{utc} to learn from the current query execution to optimize the remaining of the current query.
While this approach incurs some overheads due to cold starting for every query, the overall overheads remain negligible as it can avoid catastrophic join order choices.

ReJoin~\cite{rejoin} and Neo~\cite{neo} use recent advancements in deep reinforcement learning to generate optimal QEPs.
ReJoin uses the existing cost model of the optimizer to learn a policy network that can outperform the optimizer search strategy after more online training.
Neo uses the observed latency of QEP executions to learn a value network to predict the latency of any new QEP.
To reduce learning time and avoid choosing and evaluating prohibitively expensive join orders, Neo bootstraps the value network using latencies observed for the QEPs generated by a traditional query optimizer.

While the above systems have shown some early promising results on the ability of ML techniques to replace the traditional query optimizers, a vast number important of challenges remain open to enable practical adoption.
For example, these systems make simplifying assumptions on the grammar of supported queries, do not support physical operator selection, and in some cases assumes the availability of customized execution engines (e.g., \cite{skinnerdb}).


\subsection{Physical Database Design Automation}
One of the most important properties of  DBMSs is physical data independence.
This allows changing the physical structure of the database without requiring to change the user queries.
However, the physical database design significantly affects the performance and has to be tuned for a specific use case.
Physical database design choices include creating indices, selecting materialized views, and selecting data partitioning.
Traditionally, this has been the responsibility of database administration personnel and they have used heuristics and human judgment to perform these tasks.
Several systems have used ML techniques that learn from workload patterns either to develop decision support systems for database administrators or to automate the process.


SQL Server AutoAdmin~\cite{autoadmin} system is one of the very first systems to adopt ML techniques for the physical database design process.
It takes in a historical query workload and searches for a configuration that minimizes the cost of execution of the workload and recommends that to the database administrator.
The chosen configuration dictates which indices and materialized-views to be created and which data partitioning scheme to be used.
The search strategy uses a variant of frequent itemsets mining techniques to efficiently explore the enormous search space generated by a large number of possibilities.
The search strategy also requires estimating the cost of a new configuration without actually executing the workload.
To achieve this, AutoAdmin extends the query optimizer's cost model to support what-if queries which can assume the presence of a selected set of configurations (either hypothetical or materialized) and ignore the presence of other configurations.
However, due to the well-known limitations of the optimizer's cost model, a configuration that the optimizer thinks is better than others can be worse when implemented.
In a followup work~\cite{autoadmin_2,} AutoAdmin uses ML models trained on past QEP execution experiences to compare the execution cost of two QEPs of the same query corresponding to two different physical database configurations by formulating it as a classification problem.

SQL Server AutoAdmin~\cite{autoadmin} system is one of the very first systems to adopt ML techniques for the physical database design process.
It takes in a historical query workload and searches for a configuration that minimizes the cost of execution of the workload and recommends that to the database administrator.
The chosen configuration dictates which indices and materialized-views to be created and which data partitioning scheme to be used.
The search strategy uses a variant of frequent itemsets mining techniques to efficiently explore the enormous search space generated by a large number of possibilities.
The search strategy also requires estimating the cost of a new configuration without actually executing the workload.
To achieve this, AutoAdmin extends the query optimizer's cost model to support what-if queries which can assume the presence of a selected set of configurations (either hypothetical or materialized) and ignore the presence of other configurations.
However, due to the well-known limitations of the optimizer's cost model, a configuration that the optimizer thinks is better than others can be worse when implemented.
In a followup work~\cite{autoadmin_2,} AutoAdmin uses ML models trained on past QEP execution experiences to compare the execution cost of two QEPs of the same query corresponding to two different physical database configurations by formulating it as a classification problem.

AutoAdmin takes a reactive strategy for physical database design automation. 
It selects a static configuration based on a historical query workload and leaves it up to the database administrator to choose when to apply the change.
However, the chosen configuration may be sub-optimal for the workload in the near future.
On the contrary systems like QB5000~\cite{qb5000} and DQM~\cite{dqm} takes a proactive strategy for physical database design that completely automates the process without any intervention of a human.
QB5000 trains ML models to predict the query workload to the future and uses that to select the best set of indices.
DQM trains a deep RL model to learn a policy to opportunistically select and evict materialized views subject to storage constraints, that will have the most benefit into the future.
While these systems have shown promising initial results, much work is still needed to improve the robustness before incorporating them into enterprise systems.

\subsection{Approximate Query Processing}
While the ever-growing volumes of data enable us to glean unprecedented insights, the associated high computational and resource costs often become a bottleneck.
Approximate query processing (AQP) techniques try to mitigate this issue by generating approximate answers to the original query at a fraction of the time and cost of the original query execution.
The conventional approach to AQP is to use query time data sampling and data statistics to answer queries.
After the query is executed, the work done for that query is never reused.

Every new query execution reveals some new information about the data and executing more and more queries over time enables us to refine that information even further.
Thus, machine learning techniques provide an interesting opportunity to learn from past query executions and use that learning to approximately answer future queries.
Verdict \cite{verdict} is one of the first systems to apply this technique in the context of AQP.
It uses a data structure called \textit{Query Synopsis} to store the past query results and uses it to refine the approximate answer generated by the system for new queries.
Thus Verdict can reduce the runtime required for an approximate query for specified error bound or reduce the error bound for an approximate query with a specified time budget.
This is achieved by modeling a multivariate normal distribution model using the principle of maximum entropy.

% For the TPC-H benchmark dataset, Verdict can improve the runtime by 9x for a 4\% error bound and tighten the error bound by 85\% for a fixed time budget of 4 minutes.
% However, Verdict can only support flat select-project-join style queries with no text predicated which only covers 14 of the 21 TPC-H queries.
% Also, it cannot support updates and deletes.

Instead of learning from past queries and augmenting sampling-based methods with ML methods at execution time, DBEst~\cite{dbest} takes a pure ML-based approach for answering AQP queries.
It samples data corresponding to each predicate attribute and group by attribute values and trains regression and density models.
At execution time it uses the corresponding models and performs integration over those models to generate AQP result.

Similarly, DeepDB~\cite{deepdb} also uses ML methods to learn the joint probability distribution of the data, which can be used to answer AQP queries.
The type of the ML models used by DeepDB is called Relational Sum-Product Networks (RSPN), which belongs probabilistic graphical model family.
Given a relational database and the correlations between columns, DeepDB trains RSPNs for the tables and use the RSPN ensemble to answer SQL queries by compiling them into inference procedures over the RSPNs.















\section{ML for Physical Database Design Optimization}

\section{ML for Knob Tuning}



\section{End-to-End Learned Database Systems}

\section{Open Challenges}

\section{Conclusion}




\section{Excerpts from Papers}

Seattle Report \cite{seattle-report} Recent advances in ML have inspired our community to reflect on how
some of hard data engine problems could use ML to advance the state of the art. The most obvious such problems are in auto tuning.
For example, we can systematically replace ``magic numbers'' in database systems with data-driven learning models and use them to auto-tune system configurations.
ML also provides new hope for progress in query optimization, which has seen only minor improvements in the last two decades.
Although in principle almost any component can potentially be improved with ML, answers to some key questions are prerequisites for success, such as availability of training data, a well-thought software engineering pipeline to support an ML component (debuggability is notoriously hard), and availability of the guard-rails so that when test data or test queries deviate from the training data and training queries, the system degrades gracefully.


\bibliographystyle{unsrt}
\bibliography{main}

\end{document}