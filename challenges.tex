\section{Open Challenges}
Integrating ML methods into DBMS components has proven to optimize system performance.
Some systems have already integrated ML into enterprise DBMSs~\cite{leo, cardlearner, verdict}.
However, the field is still in its infancy and requires solving many open challenges to realize the full potential.
Next, we identify three such major open challenges:


\vspace{2mm}
\noindent \textbf{1. Worst-case Guarantees and Graceful Performance Degradation.} Integrating ML methods into internal DBMS components have proven to improve the average query execution performance.
However, ML models can have predictions that are significantly off that cause huge unexpected performance degradation.
On the contrary, traditional software components are designed to minimize the worst-case performance cost.
Worst-case performance guarantees are a crucial aspect of software systems as one single fault can have ripple effects and render the entire system unusable eventually (e.g., the evaluation time difference between a good QEP and bad QEP can be orders of magnitude).
Understanding the worst-case behavior of ML-driven software components is still an untouched area and it is possible that coming up with tight guarantees is a very hard theoretical problem.

Assuming that coming up with tight worst-case guarantees is a hard/unsolvable task, another approach to solve the same problem would be to integrate adaptive execution strategies with ML-driven components.
This requires observing the outcome of the decisions taken by ML-driven components and dynamically adjust them if the taken decision turns out to be a bad one to achieve graceful performance degradation.
Some initial work on this regard was proposed in SkinnerDB~\cite{skinnerdb} where it performs intra-query RL for choosing the best join ordering by switching between different orders and provide worst-case performance guarantees.
However, this space is still very open and much work is needed in order to make practical adoption of ML-driven components in DBMSs more pervasive.


\vspace{2mm}
\noindent \textbf{3. Rethinking DBMS Architecture}



\vspace{2mm}
\noindent \textbf{2. Enable Transfer Learning}
