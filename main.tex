%!TEX root = <main.tex>
\documentclass[acmsmall]{acmart}

% \copyrightyear{2019} 
% \acmYear{2019} 
% \setcopyright{acmcopyright}
% \acmConference[SIGMOD '19]{2019 International Conference on Management of Data}{June 30-July 5, 2019}{Amsterdam, Netherlands}
% \acmBooktitle{2019 International Conference on Management of Data (SIGMOD '19), June 30-July 5, 2019, Amsterdam, Netherlands}
% \acmPrice{15.00}
% \acmDOI{10.1145/3299869.3319874}
% \acmISBN{978-1-4503-5643-5/19/06}

\settopmatter{printacmref=false}
% \fancyhead{}

\usepackage[font=small,labelfont=bf]{caption}
\usepackage{graphicx,xspace,verbatim,comment}
\usepackage{hyperref,array,color,balance,multirow}
\usepackage{balance,float,url,amsfonts,alltt}
\usepackage{mathtools,rotating,amsmath,amssymb}
\usepackage{color,ifpdf,fancyvrb}
\usepackage{etoolbox,listings,subcaption}
\usepackage{bigstrut,morefloats,pbox}
\usepackage{amsmath}
\usepackage{algorithm}
\usepackage[noend]{algpseudocode}
\usepackage{booktabs}
\usepackage{bm}

\newtheorem{theorem}{Theorem}[section]
\newtheorem{prop}{Proposition}[section]
\newtheorem{corollary}[theorem]{Corollary}
\newtheorem{lemma}[theorem]{Lemma}
\newtheorem{definition}{Definition}[section]

\newcommand{\red}{\textcolor{red}}
\newcommand{\titlename}{Machine Learning for Database Internals: A Survey}

\DeclarePairedDelimiter{\ceil}{\lceil}{\rceil}

\newenvironment{packeditems}{
\begin{itemize}
  \setlength{\itemsep}{1pt}
  \setlength{\parskip}{0pt}
  \setlength{\parsep}{0pt}
}{\end{itemize}}

\newenvironment{packedenums}{
\begin{enumerate}
  \setlength{\itemsep}{1pt}
  \setlength{\parskip}{0pt}
  \setlength{\parsep}{0pt}
}{\end{enumerate}}

\newcolumntype{P}[1]{>{\centering\arraybackslash}p{#1}}

\DeclareMathOperator*{\argmin}{arg\,min}

\pagestyle{empty}  
\pagenumbering{arabic}

\begin{document}\sloppy
\title{\titlename}


\author{Supun Nakandala}
\affiliation{
  \institution{University of California, San Diego}
}
\email{snakanda@eng.ucsd.edu}

\maketitle
\begin{abstract}
\end{abstract}

\section{Introduction}

Machine Learning (ML) is revolutionizing many fields.
It is now being used in a diverse range of domains including e-commerce, web and social media, interactive voice agents, and even in critical applications in autonomous vehicles and healthcare.
From a software systems point of view, ML can be viewed as a flexible framework for \textit{programming the unprogrammable}.
For instance, it is nearly impossible for a software developer to program an object detector that can identify different objects from an image.
But using ML and a sufficient amount of labeled training data one can \textit{learn a program} to do the same, which can even surpass human-level accuracy.

Database management systems (DBMSs) are complex software systems, that are the result of decades of research and highly-advanced engineering efforts.
They are at the core of many critical software applications and have also influenced much of the development of other popular data management systems such as NoSQL, Big Data, and Machine Learning Systems.
As the reader may be already aware, DBMSs are also full of hard unprogrammable problems. These include query optimization, physical database design optimization, buffer management, etc., which are unprogrammable either because they have interactable search spaces and/or because of the inability to predict the future. As a result, DBMS developers have to resort to using heuristics and/or scope restrictions to solve these problems. The goal of these heuristics/restrictions is not to find an optimal solution for a specific instance, but to find a solution which has a good worst-case performance on all cases.
Thus, we survey the existing landscape of using ML to program the unprogrammable in DBMSs. 

\begin{figure*}
    \centering
    \vspace{-6mm}
    \includegraphics[height=0.85\textwidth, angle=90]{images/taxonomy.pdf}
\end{figure*}

Out of the many types of database management systems, relational database management systems (RDBMSs) remain the most widely used type.
Thus for this paper, we primarily focus on RDBMSs.
We divide the DBMS into three main sub-components: 1) Query Parser, 2) Relational Engine, and 3) Execution Engine. Some background on each of these sub-components and different ML methods is provided in Section 2.
We then identify several systems that have proposed using ML in Query Parser, Relational Engine, and Execution Engine in Section 3, 4, and 5, respectively.
In Section 6, we also identify three overarching design decisions that a DBMS developer has to make when incorporating ML into a DBMS: 1) integration mode, 2) learning source, and 3) choice of ML paradigm, and also discuss the trade-offs of available options.
A summary of where the surveyed systems fall in this taxonomy is presented in Figure 1.
While there are major accomplishments, the field is still in its infancy and many challenges remain open.
In Section 7, we identify three such open challenges: 1) improving robustness, 2) rethinking DBMS architecture, and 3) enabling transfer learning.
We give our concluding remarks in Section 8.

\section{Background}
\subsection{Database System Architecture}
\subsection{Machine Learning Methods}

\section{ML for Relational Engine}
Relational Engine is one of the most important components in a database management system,  that has been extensively studied for the last 40 years.
It takes in a query in the form of relational algebra expression and outputs a query evaluation plan, which will be evaluated by the underlying execution engine.
We identify three different sub-areas of ML applications in the relation engine: ML for (1) query optimization, (2) physical database design optimization, (3) approximate query processing.
Next, we discuss some of the most prominent work in each of these sub-areas.

\subsection{Query Optimization}

\noindent\textbf{Cardinality Estimation}

\noindent\textbf{Join Order Enumeration}

\subsection{Physical Database Design Optimization}


\subsection{Approximate Query Processing}
While the ever growing volumes of data enables us to glean unprecedented insights, the associated high computational and resource costs often becomes a bottleneck.
Approximate query processing (AQP) techniques try to mitigate this issue by generating approximate answers to the original query at a fraction of time and cost of the original query execution.
The conventional approach to AQP is to use query time data sampling and/or data statistics to answer queries.
After the query is executed the work done for that query is never reused.

Alternatively, machine learning techniques provide an interesting opportunity to learn from past query executions and use that learning to approximately answer future queries.
Every new query execution reveals some new information about the data and processing more and more queries over time reveals even more information.
Verdict \cite{verdict} is one of the very first systems to apply this technique in the context of AQP.
It uses a data structure called \textit{Query Synopsis} to store the past query results and uses it to refine the approximate answer generated by the system for every new query.
Thus Verdict is able to reduce to runtime required for an approximate query for specified error bound or reduce the error bound for an approximate query with a specified time budget.
Internally, Verdict uses the principle of maximum entropy to build a multivariate normal distribution model, which is in expectation yields superior results compared to just sampling-based techniques.
For the TPC-H benchmark dataset, Verdict can improve the runtime by 9x for a 4\% error bound and tighten the error bound by 85\% for a fixed time budget of 4 minutes.
However, Verdict can only support flat select-project-join style queries with no text predicated which only covers 14 of the 21 TPC-H queries.
Also, it cannot support updates and deletes.




\section{ML for Physical Database Design Optimization}

\section{ML for Knob Tuning}



\section{End-to-End Learned Database Systems}

\section{Open Challenges}

\section{Conclusion}




\section{Excerpts from Papers}

Seattle Report \cite{seattle-report} Recent advances in ML have inspired our community to reflect on how
some of hard data engine problems could use ML to advance the state of the art. The most obvious such problems are in auto tuning.
For example, we can systematically replace ``magic numbers'' in database systems with data-driven learning models and use them to auto-tune system configurations.
ML also provides new hope for progress in query optimization, which has seen only minor improvements in the last two decades.
Although in principle almost any component can potentially be improved with ML, answers to some key questions are prerequisites for success, such as availability of training data, a well-thought software engineering pipeline to support an ML component (debuggability is notoriously hard), and availability of the guard-rails so that when test data or test queries deviate from the training data and training queries, the system degrades gracefully.


\bibliographystyle{unsrt}
\bibliography{main}

\end{document}